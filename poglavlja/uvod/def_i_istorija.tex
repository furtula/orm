%\begin{savequote}
%    Посао аналитичког хемичара није да добије апсолутно
%    тачне резултате --- које по мом мишљењу могу бити
%    добијени само случајно --- већ да резултати које
%    добије буду што тачнији у складу са методом каја се
%    користи у анализи.
%    \qauthor{Јакоб Берелијус (1779--1848)}
%\end{savequote}
\begin{savequote}
	Прецизна и тачна мерења изгледају ненаучној популацији
	као мање узвишен и достојанствен посао од онога када
	се трага за нечим новим. Али скоро сва највећа научна
	открића су се десила захваљујући прецизним мерењима и
	стрпљивом дуготрајном раду на анализи добијених нумеричких
	резултата.
\qauthor{Барон Вилијам Томсон -- Келвин (1824--1907)}
\end{savequote}
\chapter{О мерењу}

\begin{sazetak}
  Ово поглавље обрађује следеће теме:\\[2mm]
  \begin{itemize}
  \item историјат мерења;
  \item процес мерења;
  \item значај мерења у хемијским делатностима.
  \end{itemize}
\end{sazetak}

\section{Почеци мерења у хемији}

\section{Мерење}

Данас се у развијеном свету одвајају велике суме новца за мерења
у природним наукама. То је разумљиво с обзиром да добијени резултати
увелико помажу у решавању проблема са којима се сусреће човечанство.
Иако се људи срећу са различитим врстама мерења свакодневно, мало
је оних који би могли правилно дефинисати мерење. Да би се оно
дефинисало потребно је описати и термине \emph{величина} и
\emph{вредност} величине.

\begin{df}[Величина]{def:velicina}
  Особина неког тела, супстанце или процеса која се недвосмислено
  разликује од осталих особина и која се може квантитативно одредити,
  назива се величина.
\end{df}

\begin{df}[Вредност]{def:vrednost}
  Димензија неке величине која је углавном представљена мерним јединицама
  помноженим са неким бројем представља вредност те величине.
\end{df}

Имајући на уму дефиниције \ref{def:velicina} и \ref{def:vrednost} може
се дати и јасан опис мерења.

\begin{df}[Мерење]{def:merenje}
  Скуп операција са циљем одређивања \emph{вредности} (углавном
  нумеричке) неке \emph{величине} се назива мерењем.
\end{df}
Поменуте операције у дефиницији \ref{def:merenje} су:
\begin{itemize}[label={$\RHD$}]
\item Дефинисање проблема;
\item одабир величина које ће помоћи у одговору на проблем;
\item избор начина којим ће се добити вредности величина;
\item спровођење мерења и прикупљање и статистичка обрада
добијених резултата;
\item давање одговора на дати проблем у складу са добијеним
резултатима мерења.
\end{itemize}

Често се у пракси погрешно представљају резултати мерења као одговор
на дати проблем. Наиме, хемијско мерење углавном може дати само делимичан
одговор на дати проблем.
\begin{pr}[]{pr:merenje}
  Утврђивање леталне количине једињења арсена у желудачном садржају
  преминулог говори ,,само'' о узроку његове смрти, али \emph{никако}
  о кривици осумњиченог.
\end{pr}

\begin{thebibliography}{9}
\bibitem{BHJG06} D. Brynn Hibbert, J. Justin Gooding, \textit{Data Analysis
  for Chemistry --- An Introductory Guide for Students and Laboratory
  Scientists\/}, Oxford Univ. Press, Oxford, 2006.
\end{thebibliography}
