%\begin{savequote}
%    Посао аналитичког хемичара није да добије апсолутно
%    тачне резултате --- које по мом мишљењу могу бити
%    добијени само случајно --- већ да резултати које
%    добије буду што тачнији у складу са методом каја се
%    користи у анализи.
%    \qauthor{Јакоб Берелијус (1779--1848)}
%\end{savequote}
\begin{savequote}
	Прецизна и тачна мерења изгледају ненаучној популацији
	као мање узвишен и достојанствен посао од онога када
	се трага за нечим новим. Али скоро сва највећа научна
	открића су се десила захваљујући прецизним мерењима и
	стрпљивом дуготрајном раду на анализи добијених нумеричких
	резултата.
\qauthor{Барон Вилијам Томсон -- Келвин (1824--1907)}
\end{savequote}
\chapter{О мерењу}

\begin{sazetak}
  Истраживања у природним наукама су заснована на мерењима.
  Ово поглавље ће обрадити следеће теме:\\[2mm]
  \begin{itemize}
  \item историјат мерења;
  \item процес мерења;
  \item значај мерења у хемијским делатностима.
  \end{itemize}
\end{sazetak}

\section{Почеци мерења у хемији}